\documentclass[11pt]{article}
\usepackage{graphicx}
\usepackage{mathtools}
\begin{document}

\begin{titlepage}
	\begin{center}
		\vspace*{\fill}
		\begin{Huge}
			Homework \#3\\
			
			
			Web Information Retrieval
		\end{Huge}
		\newline
		\newline
		Federico Arcangeli \-- 1771113, Fabrizio Tropeano \--1771734\\
		\vspace*{\fill}	
	\end{center}
\end{titlepage}


\section{Introduction}
The following document concerns the text classification problems, solved by applying supervised learning techniques and is part of the third homework of the \textit{Web Information Retrieval} course.\\
The goal of this homework is to tune and train different classifiers. In order to complete the job we used the \textit{scikit-learn} library, to adjust the classifiers, and \textit{nltk} library to compute the lexical analysis.
The homework is divided in two parts: 
\begin{itemize}
\item{\textbf{Part One:} } In this part of the homework we tune and train a classifier based on \textit{K-Nearest-Neighbors} technique for detecting spam comments associated to YouTube videos.
\item{\textbf{Part Two:} } Here, we tune and train three different classifiers to solve a sentiment analysis problem.
\end{itemize}




\section{Part One}
In this part of the homework we had to solve a Spam-Ham classification problem, using a kNN classifier and some scikit-learn tools, trying to find the best combination of parameters for solving the problem. In order to reach the goal and to avoid overfitting, we performed a \textit{10-Fold-Cross-Validation process} through the grid-searchCV, a tool able to exhaustively considers all parameter combinations. To represent documents as vectors we used the scikit-learn class \textit{TfIdfVectorizer}.  Once we have the best combination of parameters, we started to evaluate the performance of the classifier on the Test-Set.  \\
For the evaluation process we used the following tools provided by scikit-learn:
\begin{itemize}
\item{$\textbf{metrics.classification\_report:}$} To have a report with the main classification metrics
\item{$\textbf{metrics.confusion\_matrix:}$} To compute the Confusion-Matrix, a tool to evaluate the accuracy of a classification. By definition a confusion matrix $C$ is such that $C_{i, j}$ is equal to the number of observations known to be in group $i$ but predicted to be in group $j$. 
\item{$\textbf{metrics.accuracy\_score:}$} Used to calculate the Normalized-accuracy. This function computes subset accuracy, i.e. the set of labels predicted for a sample must exactly match the corresponding set of ground-truth labels. 
\item{$\textbf{metrics.matthews\_corrcoef:}$} In order to compute the Matthews correlation coefficient. The Matthews correlation coefficient is used in machine learning as a measure of the quality of binary classifications. It takes into account true and false positives and negatives. The MCC is in essence a correlation coefficient value between -1 and +1. A coefficient of +1 represents a perfect prediction, 0 an average random prediction and -1 an inverse prediction. 
\end{itemize}
To tune and train the classifier, we used the data collected inside the directory \textit{Training} while to evaluate the performance of the final classifier, we used the data collected inside the directory \textit{Test}.\\

\subsection{Conclusions}
\subsubsection{Dataset Part One}
	\begin{tabular}{|l|c|r|}
	\hline
 		& Training Set & Test Set\\
	\hline
		Spam & 122 & 53\\
	\hline
		Ham & 120 & 52\\
	\hline
	\end{tabular}
	
\subsubsection{List of all parameters}
\begin{itemize}

		\item{\textbf{Tokenizer:}} Tokenization is the process of classifying sections of a string of input characters. Before creating the tokenizer we used the 		English stemmer with and without \textit{stopwords}. 
		\begin{itemize}
	\item{$\textbf{Ngram\_range:}$}  An n-gram is a contiguous sequence of $n$ items from a given sequence of text. We analyzed the data once as unigrams 		(n=1) and once as bigrams (n=2) .\\
		For example, with a sentence like \textit{"The quick brown fox jumps over the lazy dog"} we have :\\
		\textbf{Unigram:} "The", "quick", "brown", "fox", "jumps", "over", "the", "lazy", "dog"\\
		\textbf{Bigram:} "The quick", "quick brown", "brown fox", "fox jumps", "jumps over", "over the", "the lazy", "lazy dog"
	
	\end{itemize}


\item{\textbf{K-Nearest-Neighbors}} 

	\begin{itemize}

		\item{\textbf{weights:}} A weight function used in prediction. We used \textit{uniform} and \textit{distance}. In 			uniform, all points are 				weighted equally while in distance closer neighbors of a query point will have a 		greater influence than neighbors which are further 				away.

		\item{$\textbf{n\_neighbors:}$}  It is desirable for $n$ to be odd to make ties less likely. The values 		3 and 5 are the most common choices. 				However, we tried values in a range from 3 to 11.	
	\end{itemize}

\end{itemize}

\subsection{How to perform the training-validation process using more than one CPU core}
\includegraphics[scale=0.8]{gridsearch.png}\\


Defining the Grid-SearchCV, we used a Pipeline object to assemble several steps that can be cross-validated while setting different parameters. 
These parameters refer mostly on the classifiers and the \textit{Tf-Idf Vectorizer}. We set the rest of parameters as follows:
\begin{itemize}
\item{\textbf{scoring:}} We set the scoring function on the Matthews Correlation Coefficient. 
\item{\textbf{cv:}} We set the value of cv to be equal to 10 since in the homework specification, it was requested to perform a 10-Fold-Cross-Validation.
\item{$\textbf{n\_jobs:}$} This is the number of jobs to run in parallel. We have to set this parameter to a value equal to the number of core in the machine. 
\item{\textbf{verbose:}} It controls the verbosity. The higher, the more messages. 

\end{itemize}

\subsection{The best configuration of parameters found by the GridSearchCV object at the end of the
      10-Fold-Cross-Validation process.}

At the end of the execution we found this as the best configuration of parameters : \\
\includegraphics{bestconfiguration1.png}


\subsection{Report}
\includegraphics{report1.png}

\subsection{Confusion Matrix}
\includegraphics{confusionmatrix1.png}

\subsection{Normalized Accuracy}
\includegraphics{accuracy1.png}

\subsection{Matthews Correlation Coefficient}
\includegraphics{mcc1.png}





\section{Part Two}
In this part of the homework we had to tune and train three different classifiers to solve a sentimental analysis problem. These classifiers have to be able to classify sentences representing positive and negative opinions about movies. We used only supervised learning methods: \textit{K-Nearest Neighbors}, \textit{Multinomial Naive Bayes Classifier} and \textit{Support Vector Machine}. We discussed K-Nearest Neighbors approach in the previous section. Now, we see two other methods. 

\subsection{Multinomial Naive Bayes}
The multinomial naive bayes model is a probabilistic supervised learning method. The probability of a document $d$ being in class $c$ is computed as: \\
\begin{center}
{$\Pr{(c|d)} \propto{\Pr{(c)} \prod_{1\leq{k}\leq{n_{d}}} \Pr{(t_{k} | c)}}$}
\end{center}
where $\Pr{(t_{k} | c)}$ is the conditional probability of term $t_{k}$ occurring in a document of class $c$. We interpret $\Pr{(t_{k} | c)}$ as a measure of how much evidence $t_{k}$ contributes that $c$ is the correct class. $\Pr{(c)}$ is the prior probability of a document occurring in class $c$. If a document's terms do not provide clear evidence for one class versus another, we choose the one that has a higher prior probability. In text classification, our goal is to find the best class for the document. The best class in $NB$ classification is the most likely or maximum a posteriori (MAP) class $c_{map}$: 
\begin{center}
$c_{map} = argmax_{c \in{C}} \hat{Pr} (c) \prod_{1\leq{k}\leq{n_{d}}} \hat{Pr} (t_{k} | c)$
\end{center}
We have $\hat{Pr}$ instead of $\Pr$ because we do not know the true values of the parameters $\Pr{(c)}$ and $\Pr{(t_{k} | c)}$, but estimate them. 



\subsection{Support Vector Machine}
Support vector machines are supervised learning models with associated learning algorithms that analyze data used for classification. An SVM model is a representation of the examples as points in space, mapped so that the examples of the separate categories are divided by a clear gap that is as wide as possible. New examples are then mapped into that same space and predicted to belong to a category based on which side of the gap they fall. 

\subsection{Conclusions}

\subsubsection{Dataset}
\begin{tabular}{|l|c|r|}
\hline
 & Training Set & Test Set\\
\hline
Positive & 308 & 308\\
\hline
Negative & 249 & 250\\
\hline
\end{tabular}

\subsubsection{List of all parameters}


\begin{itemize}

\item{\textbf{Vectorizer}}
	\begin{itemize}

		\item{\textbf{Tokenizer:}} Tokenization is the process of classifying sections of a string of input characters. Before creating the tokenizer we used the 		English stemmer with and without \textit{stopwords}. 

	\item{$\textbf{Ngram\_range:}$}  An n-gram is a contiguous sequence of $n$ items from a given sequence of text. We analyzed the data once as unigrams 		(n=1) and once as bigrams (n=2) .\\
		For example, with a sentence like \textit{"The quick brown fox jumps over the lazy dog"} we have :\\
		\textbf{Unigram:} "The", "quick", "brown", "fox", "jumps", "over", "the", "lazy", "dog"\\
		\textbf{Bigram:} "The quick", "quick brown", "brown fox", "fox jumps", "jumps over", "over the", "the lazy", "lazy dog"
	
	\end{itemize}


\item{\textbf{K-Nearest-Neighbors}} 

	\begin{itemize}

		\item{\textbf{weights:}} A weight function used in prediction. We used \textit{uniform} and \textit{distance}. In 			uniform, all points are 				weighted equally while in distance closer neighbors of a query point will have a 		greater influence than neighbors which are further 				away.

		\item{$\textbf{n\_neighbors:}$}  It is desirable for $n$ to be odd to make ties less likely. The values 		3 and 5 are the most common choices. 				However, we tried values in a range from 3 to 11.	
	\end{itemize}

	
\item{\textbf{Naive Bayes Classifier}}

	\begin{itemize}

		\item{\textbf{alpha:}} It is the smoothing factor, that is used to eliminate zeros. It is obtained as estimation because of sparseness, and simply adds 		one to each count. We used the following values for $\alpha = [.001,.01,1.0]$.

		\item{$\textbf{fit\_prior:}$} It is a boolean value chosen in order to learn class prior probabilities or not. If false a uniform prior will be used. 
	\end{itemize}


\item{\textbf{Support Vector Machine}}

	\begin{itemize}
		\item{$\textbf{C:}$} The parameter C is a regularization term, which provides a way to control overfitting. It is a trade off on how fat it can make the 			margin with respect to how many points have to be moved around to allow this margin. We tried the following value : 1,10,100

		\item{$\textbf{class\_weight:}$} Set the parameter $C$ of class $i$ to $class\_weight[i] * C$

		\item{$\textbf{kernel:}$} A kernel function K is such a function that corresponds to a dot product in some expanded feature space. We tried out three 			variants of such function: linear kernel, polynomial kernel and radial basis functions $(RBF)$. 	
	\end{itemize}
\end{itemize}

\subsection{Best configuration of parameters found by the Grid-SearchCV object at the end of the 10-Fold-Cross-Validation process}
\subsubsection{K-Nearest Neighbors}
\includegraphics{knnperformance}
\subsubsection{Naive Bayes Classifier}
\includegraphics{nbcperformance}
\subsubsection{Support Vector Machine}
\includegraphics{svmperformance}

\subsection{Report, Confusion Matrix, Normalized Accuracy, Matthews Correlation Coefficient}
\subsubsection{K-Nearest Neighbors}
\includegraphics{knnfinal.png}
\subsubsection{Naive Bayes Classifier}
\includegraphics{nbcfinal.png}
\subsubsection{Support Vector Machine}
\includegraphics{svmfinal.png}


\end{document}
